\documentclass[]{article}
\usepackage[portrait, margin=5mm]{geometry}
\usepackage{amsmath}
\usepackage{amssymb}
\usepackage{amsfonts}
\usepackage{listings}
\pagenumbering{gobble}

\renewcommand{\.}{\hskip .75pt}

\renewcommand{\arraystretch}{1.25}

\newcommand{\fin}[1]{[\.#1\.]}

\DeclareMathOperator{\aand}{\;\wedge\;}
\DeclareMathOperator{\st}{,\;}
\DeclareMathOperator{\ex}{\,\exists}

\let\r=\rightarrow
\let\*=\cdot

\begin{document}

\lstset{
	basicstyle=\ttfamily\small,
	literate=
	{→}{{$\rightarrow$}}1
	{∀}{{$\forall$}}1
	{∃}{{$\exists$}}1
	{×}{{$\times$}}1
	{σ}{{$\sigma$}}1
	{τ}{{$\tau$}}1
	{α}{{$\alpha$}}1
	{≠}{{$\neq$}}1
	{≤}{{$\leq$}}1
	{≥}{{$\geq$}}1
	{↔}{{$\iff$}}1
	{¬}{{$\neg$}}1
	{∧}{{$\wedge$}}1
	{•}{$\bullet$}1
	{·}{$\cdot$}1
	{⬝}{$\cdot$}1
	{ℕ}{{$\mathbb{N}$}}1
	{ℤ}{{$\mathbb{Z}$}}1
	{ₗ}{{$_l$}}1
	{₀}{{$_0$}}1
	{∑}{{$\sum$}}1
	{ᵀ}{{$^\texttt{T}$}}1
	{ᵥ}{{$_v$}}1
	{⁻¹}{{$^{-1}$}}1
}


\title{Duality theory in linear optimization and its extensions\,---\,formally verified}
\date{}
\maketitle


\noindent \textbf{Abstract:}\;
Farkas established that a system of linear inequalities has a solution if and only if we cannot obtain
a contradiction by taking a linear combination of the inequalities.
We state and formally prove several Farkas-like theorems in Lean 4.
Furthermore, we consider a linearly ordered field extended with two special elements denoted by $\bot$ and $\top$
where $\bot$ is below every element and $\top$ is above every element.
We define $\bot + a = \bot = a + \bot$ of all $a$ and we define $\top + b = \top = b + \top$ for all $b \neq \bot$.
For multiplication, we define $\bot \cdot c = \bot = c \cdot \bot$ for every $c \ge 0$ but
$\top \cdot d = \top = d \cdot \top$ only for $d > 0$ because $\top \cdot 0 = 0 = 0 \cdot \top$.
We extend certain Farkas-like theorems to a setting where coefficients are from an extended linearly ordered field.


\section{Introduction}

Fredholm (TODO citation) established that
a system of linear equalities has a solution if and only if
we cannot obtain a contradiction by taking a linear combination of the equalities.
We state their theorem as follows.

\medskip \noindent
\textbf{Theorem (equalityFredholm):}
Let $I$ and $J$ be finite types.
Let $F$ be a linearly ordered field.
Let $A$ be a matrix of type $(I \times J) \r F$.
Let $b$ be a vector of type $I \r F$.
Exactly one of the following exists:
\begin{itemize}
\item vector $x : J \r F$ such that $A \* x = b$
\item vector $y : I \r F$ such that $A^T\! \* y = 0$ and $b \* y \neq 0$
\end{itemize}
Geometric interpretation of equalityFredholm is straightforward.
The column vectors of $A$ generate a hyperplane in the
$|I|$-dimensional Euclidean space that goes contains the origin.
The point $b$ either lies in this hyperplane (in this case, the entries
of $x$ give coefficients which, when applied to the column vectors of $A$,
give a vector from the origin to the point $b$),
or there exists a line through the origin that is orthogonal to
all the column vectors of $A$ (i.e., orthogonal to the entire hyperplane)
such that $b$ projected onto this line falls outside of the origin
(in this case, $y$ gives a direction of this line), i.e., to a different point
from where all column vectors of $A$ get projected.

This theorem can be given in much more general settings.
In our paper, however, this is the only version we provide.
This staple of linear algebra is not our main focus
but a byproduct of the other theorems we prove.
In particular, we obtain equalityFredholm as an immediate corollary
of the following theorem.

\medskip \noindent
\textbf{Theorem (equalityFredholm\_lt):}
Let $I$ and $J$ be finite types.
Let $F$ be a linearly ordered field.
Let $A$ be a matrix of type $(I \times J) \r F$.
Let $b$ be a vector of type $I \r F$.
Exactly one of the following exists:
\begin{itemize}
\item vector $x : J \r F$ such that $A \* x = b$
\item vector $y : I \r F$ such that $A^T\! \* y = 0$ and $b \* y < 0$
\end{itemize}
The way we state the theorem exemplifies certain patterns that
permeate through our work. Our results are phrased as
``there are two systems of (in)equalities; exactly one of them has a solution''.
This goes hand-in-hand with our decision to focus mostly on ``symmetric'' Farkas-like theorems.
Note that it must be impossible to satisfy the second system by $y=0$.
Had it been allowed, we would have said nothing about the first system as it would
have to lead to a contradiction every time. The constraint $b \* y < 0$ disqualifies
the zero solution in most of our theorems. Intuitively, it should make sense that
one of the systems is always ``strict'' (to easily see why, consider $I$ and $J$ singletons),
which also means that our two systems will never be ``fully symmetric''.

Farkas (TODO citation) gave a similar characterization for systems of linear equalities
with nonnegative variables.
We state his theorem as follows.

\medskip \noindent
\textbf{Theorem (equalityFarkas):}
Let $I$ and $J$ be finite types.
Let $F$ be a linearly ordered field.
Let $A$ be a matrix of type $(I \times J) \r F$.
Let $b$ be a vector of type $I \r F$.
Exactly one of the following exists:
\begin{itemize}
\item nonnegative vector $x : J \r F$ such that $A \* x = b$
\item vector $y : I \r F$ such that $A^T\! \* y \ge 0$ and $b \* y < 0$
\end{itemize}
Geometric interpretation of equalityFarkas is easy.
The column vectors of $A$ generate a cone in the
$|I|$-dimensional Euclidean space from the origin towards some infinity.
The point $b$ either lies inside this cone (in this case, the entries
of $x$ give nonnegative coefficients which,
when applied to the column vectors of $A$,
give a vector from the origin to the point $b$),
or there exists a hyperplane that contains the origin and that
strictly separates $b$ from given cone
(in this case, $y$ gives a normal vector of this hyperplane).

We prove equalityFredholm\_lt by applying equalityFarkas to the matrix $(A~|-\!\! A)$.
However, equalityFarkas will be proved later, from a more general theorem.

Minkowski (TODO citation) similarly established that
a system of linear inequalities has a nonnegative solution if and only if
we cannot obtain a contradiction by taking a nonnegative linear combination of the inequalities.
For reasons that will be apparent later, we give two versions of this theorem.

\medskip \noindent
\textbf{Theorem (inequalityFarkas):}
Let $I$ and $J$ be finite types.
Let $F$ be a linearly ordered field.
Let $A$ be a matrix of type $(I \times J) \r F$.
Let $b$ be a vector of type $I \r F$.
Exactly one of the following exists:
\begin{itemize}
\item nonnegative vector $x : J \r F$ such that $A \* x \le b$
\item nonnegative vector $y : I \r F$ such that $A^T\! \* y \ge 0$ and $b \* y < 0$
\end{itemize}
Geometric interpretation of inequalityFarkas is a bit harder.
The column vectors of $A$ generate a cone in the
$|I|$-dimensional Euclidean space from the origin to some infinity.
The point $b$ determines an orthogonal cone that starts in $b$ and goes to
negative infinity in the direction of all coordinate axes.
Either these two cones intersect (in this case, the entries
of $x$ give nonnegative coefficients which,
when applied to the column vectors of $A$,
give a vector from the origin to a point in the intersection),
or there exists a hyperplane that contains the origin and that
strictly separates $b$ from the cone generated by $A$ but
does not cut through the positive orthant, i.e., the origin
is the only nonnegative point contained in the hyperplane
(in this case, $y$ gives a normal vector of this hyperplane).

\medskip \noindent
\textbf{Theorem (inequalityFarkas\_neg):}
Let $I$ and $J$ be finite types.
Let $F$ be a linearly ordered field.
Let $A$ be a matrix of type $(I \times J) \r F$.
Let $b$ be a vector of type $I \r F$.
Exactly one of the following exists:
\begin{itemize}
\item nonnegative vector $x : J \r F$ such that $A \* x \le b$
\item nonnegative vector $y : I \r F$ such that $(-A^T) \* y \le 0$ and $b \* y < 0$
\end{itemize}
Obviously, inequalityFarkas\_neg is an immediate corollary of inequalityFarkas.
We prove inequalityFarkas by applying equalityFarkas 
to the matrix $(1~|~A)$ where $1$ is the identity matrix of type
$(I \times I) \r F$.

The next theorem generalizes equalityFarkas to structures where
multiplication does not have to be commutative.
Furthermore, it supports infinitely many equations.

\medskip \noindent
\textbf{Theorem (coordinateFarkas):}
Let $I$ be any type.
Let $J$ be a finite type.
Let $R$ be a linearly ordered division ring.
Let $A$ be an $R$-linear map from from $(I \r R)$ to $(J \r R)$.
Let $b$ be an $R$-linear map from from $(I \r R)$ to $R$.
Exactly one of the following exists:
\begin{itemize}
\item nonnegative vector $x : J \r R$ such that, for all $w : I \r R$, we have
$ \sum_{j : J}\; (A~w)_j \bullet x_j = b~w $
\item vector $y : I \r R$ such that $A~y \ge 0$ and $b~y < 0$
\end{itemize}
In the next generalization, we replace the partially ordered module $I \r R$ by
a general $R$-module $W$.

\medskip \noindent
\textbf{Theorem (scalarFarkas):}
Let $J$ be a finite type.
Let $R$ be a linearly ordered division ring.
Let $W$ be an $R$-module.
Let $A$ be an $R$-linear map from from $W$ to $(J \r R)$.
Let $b$ be an $R$-linear map from from $W$ to $R$.
Exactly one of the following exists:
\begin{itemize}
\item nonnegative vector $x : J \r R$ such that, for all $w : W$, we have
$ \sum_{j : J}\; (A~w)_j \bullet x_j = b~w $
\item vector $y : W$ such that $A~y \ge 0$ and $b~y < 0$
\end{itemize}
In the most general theorem, stated below, we replace certain occurrences of $R$ by
a linearly ordered $R$-module $V$ whose order respects order on $R$.
This result origins from TODO.

\medskip \noindent
\textbf{Theorem (fintypeFarkasBartl):}
Let $J$ be a finite type.
Let $R$ be a linearly ordered division ring.
Let $W$ be an $R$-module.
Let $V$ be a linearly ordered $R$-module\footnote{We furthermore require 
monotonicity of scalar multiplication by nonnegative elements on the left.
This assumption will be implicit in later occurrences.}.
Let $A$ be an $R$-linear map from $W$ to $(J \r R)$.
Let $b$ be an $R$-linear map from $W$ to $V$.
Exactly one of the following exists:
\begin{itemize}
\item nonnegative vector family $x : J \r V$ such that, for all $w : W$, we have
$ \sum_{j : J}\; (A~w)_j \bullet x_j = b~w $
\item vector $y : W$ such that $A~y \ge 0$ and $b~y < 0$
\end{itemize}
In the last branch, $A~y \ge 0$ uses the partial order\footnote{The order on $(J \r R)$ is
always the coordinate-wise application of $R$'s linear order.} on $(J \r R)$ whereäs
$b~y < 0$ uses the linear order\footnote{In case $V$ has finite dimension, you can choose
an arbitrary direction and project vectors from $V$ onto it, or you can order elements of
$V$ lexicographically.} on $V$.
Note that fintypeFarkasBartl subsumes scalarFarkas (as well as the other versions based on equality),
since $R$ can be viewed as a linearly ordered module over itself.
We prove fintypeFarkasBartl in Section TODO, which is where the heavy lifting comes.

Until now, we have talked about known results.
What follows is a new extension of the theory.

\medskip \noindent
\textbf{Definition:}
Let $F$ be a linearly ordered field.
We define an \textbf{extended} linearly ordered field $F_\infty$ as
$F \cup \{ \bot, \top \}$ with the following properties.
Let $p$ and $q$ be numbers from $F$.
We have $\bot < p < \top$.
We define addition, scalar action, and negation on $F_\infty$ as follows:
\begin{center}
	\begin{tabular}{ c || c | c | c | }
		+ & $\bot$ & $q$ & $\top$  \\
		\hline\hline
		$\bot$ & $\bot$ & $\bot$ & $\bot$  \\ 
		\hline
		$p$ & $\bot$ & $p\!+\!q$ & $\top$  \\ 
		\hline
		$\top$ & $\bot$ & $\top$ & $\top$ \\ 
		\hline
	\end{tabular}
	\qquad\qquad\qquad
	\begin{tabular}{ c || c | c | c | }
		$\bullet$ & $\bot$ & $q$ & $\top$  \\
		\hline\hline
		$0$ & $\bot$ & $0$ & $0$  \\ 
		\hline
		$p>0$ & $\bot$ & $p \cdot q$ & $\top$  \\ 
		\hline
	\end{tabular}
	\qquad\qquad\qquad
	\begin{tabular}{ c || c | c | c }
	$-$ & $\bot$ & $q$ & $\top$  \\
	\hline\hline
	& $\top$ & $-q$ & $\bot$  
	\end{tabular}
\end{center}
Informally speaking, $\top$ represents the positive infinity,
$\bot$ represents the negative infinity, and we say that
$\bot$ is ``stronger'' than $\top$ in~all arithmetic operations.
The surprising parts are $\bot + \top = \bot$ and $0 \.\bullet \bot = \bot$.
Because of that, $F_\infty$ is not a field.
However, $F_\infty$ is still a densely linearly ordered abelian monoid
with characteristic zero.

\medskip \noindent
\textbf{Theorem (extendedFarkas):}
Let $I$ and $J$ be finite types.
Let $F$ be a linearly ordered field.
Let $A$ be a matrix of type $(I \times J) \r F_\infty$.
Let $b$ be a vector of type $I \r F_\infty$.
Assume that $A$ does not have $\bot$ and $\top$ in the same row.
Assume that $A$ does not have $\bot$ and $\top$ in the same column.
Assume that $A$ does not have $\top$ in any row where $b$ has $\top$.
Assume that $A$ does not have $\bot$ in any row where $b$ has~$\bot$.
Exactly one of the following exists:
\begin{itemize}
\item nonnegative vector $x : J \r F$ such that $A \* x \le b$
\item nonnegative vector $y : I \r F$ such that $(-A^T) \* y \le 0$ and $b \* y < 0$
\end{itemize}
Note that extendedFarkas looks pretty much like equalityFarkas\_neg and,
in certain sense, generalizes it. Indeed, in Section TODO, we prove
extendedFarkas using equalityFarkas\_neg and some additional machinery.

Next we define an extended notion of linear program, i.e.,
linear programming over extended linearly ordered fields.
The implicit intention is that the linear program is to be minimized.

\medskip \noindent
\textbf{Definition:}
Let $I$ and $J$ be finite types.
Let $F$ be a linearly ordered field.
Let $A$ be a matrix of type $(I \times J) \r F_\infty$.
Let $b$ be a vector of type $I \r F_\infty$.
Let $c$ be a vector of type $J \r F_\infty$.
Assume that $A$ does not have $\bot$ and $\top$ in the same row.
Assume that $A$ does not have $\bot$ and $\top$ in the same column.
Assume that $b$ does not contain $\bot$.
Assume that $c$ does not contain $\bot$.
Assume that $A$ does not have $\top$ in any row where $b$ has $\top$.
Assume that $A$ does not have $\bot$ in any column where $c$ has $\top$.
We say that $P = (A, b, c)$ is a \textbf{linear program} over $F_\infty$.
We say that its \textbf{constraints} are indexed by $I$.
We say that its \textbf{variables} are indexed by $J$.
We say that a nonnegative vector $x : J \r R$ is
a \textbf{solution} to $P$ if and only if $A \* x \le b$.
We say that $P$ \textbf{reaches} an objective value $r$
if and only if there exists $x$ such that $x$ is a solution to $P$
and $c \* x = r$.
We say that $P$ is \textbf{feasible} if and only if $P$ reaches a finite\footnote{
It would be perhaps more natural to say that $P$ reaches a value different
from $\top$. However, since $\bot$ cannot be reached because of the way
linear programming is defined, it is equivalent to our definition by
reaching a finite value.} value.
We say that $P$ is \textbf{bounded by} a finite value $r$ if and only if,
for every value $p$ reached by $P$, we have $r \le p$.
We say that $P$ is \textbf{unbounded} if and only if there is no finite value $r$
such that $P$ is bounded by $r$.
We say that the linear program $(-A^T, c, b)$ is the \textbf{dual} of $P$.

\medskip \noindent
\textbf{Theorem (weakDuality):}
Let $F$ be a linearly ordered field.
Let $P$ be a linear program over $F_\infty$.
If $P$ reaches $p$ and the dual of $P$ reaches $q$,
then $p + q \ge 0$.

\medskip \noindent
\textbf{Definition:}
Let $F$ be a linearly ordered field.
Let $P$ be a linear program over $F_\infty$.
We define the \textbf{optimum} of $P$ as follows.
If $P$ is feasible and unbounded, its optimum is $\bot$.
If $P$ is not feasible, its optimum is $\top$.
In all other cases, we ask whether $P$ reaches a finite value $f$ such that
$P$ is bounded by $f$. If so, its optimum is $f$.
Otherwise, $P$ does not have optimum.\footnote{By the end of the paper, we will
have proved that optimum always exists, i.e., it cannot be the case that the set of
objective values reached by $P$ has a finite infimum that is not attained.
However, because we do not have the theorem now, the optimum is defined as a partial function
from linear programs to $F_\infty$.}

\medskip \noindent
\textbf{Theorem (strongDuality):}
Let $F$ be a linearly ordered field.
Let $P$ be a linear program over $F_\infty$.
If $P$ or its dual is feasible (at least one of them),
then $P$ has optimum $p$ and the dual of $P$ has optimum $q$ such that
$p = -q$ as defined for $F_\infty$.

\section{Formalization}

\subsection{We start with a review of algebraic typeclasses that our project depends on}

Additive semigroup is a structure on any type with addition where the addition is associative:
\begin{lstlisting}
class AddSemigroup (G : Type u) extends Add G where
  add_assoc : ∀ a b c : G, (a + b) + c = a + (b + c)
\end{lstlisting}
Additive monoid is an additive semigroup with the zero element, thanks to which we can
define a scalar multiplication by the natural numbers:
\begin{lstlisting}
class AddMonoid (M : Type u) extends AddSemigroup M, AddZeroClass M where
  nsmul : ℕ → M → M
  nsmul_zero : ∀ x : M, nsmul 0 x = 0 
  nsmul_succ : ∀ (n : ℕ) (x : M), nsmul (n + 1) x = nsmul n x + x 
\end{lstlisting}
Subtractive monoid is an additive monoid that adds two more operations (unary and binary minus)
that satisfy some basic properties:
\begin{lstlisting}
class SubNegMonoid (G : Type u) extends AddMonoid G, Neg G, Sub G where
  sub := SubNegMonoid.sub'
  sub_eq_add_neg : ∀ a b : G, a - b = a + -b 
  zsmul : ℤ → G → G
  zsmul_zero' : ∀ a : G, zsmul 0 a = 0 
  zsmul_succ' (n : ℕ) (a : G) : zsmul (Int.ofNat n.succ) a = zsmul (Int.ofNat n) a + a
  zsmul_neg' (n : ℕ) (a : G) : zsmul (Int.negSucc n) a = -(zsmul n.succ a)
\end{lstlisting}
Additive group is a subtractive monoid in which the unary minus acts as an inverse with respect to addition:
\begin{lstlisting}
class AddGroup (A : Type u) extends SubNegMonoid A where
  add_left_neg : ∀ a : A, -a + a = 0
\end{lstlisting}
Abelian group is defined as an additive group that is a commutative additive monoid at the same time (TODO AddCommMonoid):
\begin{lstlisting}
class AddCommGroup (G : Type u) extends AddGroup G, AddCommMonoid G
\end{lstlisting}
Ring is defined as a semiring that is an abelian group at the same time and has 1 that behaves well (TODO Semiring):
\begin{lstlisting}
class Ring (R : Type u) extends Semiring R, AddCommGroup R, AddGroupWithOne R
\end{lstlisting}
Division ring is a ring with a lot of extra requirements (TODOs DivInvMonoid, Nontrivial, NNRatCast, RatCast):
\begin{lstlisting}
class DivisionRing (α : Type*) extends Ring α, DivInvMonoid α, Nontrivial α, NNRatCast α, RatCast α where
  mul_inv_cancel : ∀ (a : α), a ≠ 0 → a * a⁻¹ = 1
  inv_zero : (0 : α)⁻¹ = 0
  nnratCast := NNRat.castRec
\end{lstlisting}
We define a linearly ordered division ring as a division ring that is a linearly ordered ring at the same time (TODO all about order):
\begin{lstlisting}
class LinearOrderedDivisionRing (R : Type*) extends LinearOrderedRing R, DivisionRing R
\end{lstlisting}
Linearly ordered field is defined as a linearly ordered commutative ring that is a field at the same time (TODO Field):
\begin{lstlisting}
class LinearOrderedField (α : Type*) extends LinearOrderedCommRing α, Field α
\end{lstlisting}
Note that LinearOrderedDivisionRing is not a part of the algebraic hierarchy provided by Mathlib,
hence LinearOrderedField does not inherit LinearOrderedDivisionRing, thus we provide a custom
instance that converts LinearOrderedField to LinearOrderedDivisionRing:
\begin{lstlisting}
instance LinearOrderedField.toLinearOrderedDivisionRing {F : Type*} [instF : LinearOrderedField F] :
  LinearOrderedDivisionRing F := { instF with }
\end{lstlisting}
This instance is needed for the step from coordinateFarkas to equalityFarkas.

\subsection{Vectors and stuff}

We distinguish two types of vectors; implicit vectors and explicit vectors.
Implicit vectors are members of a vector space; they don't have any internal structure.
Explicit vectors are functions from coordinates to values.
The set of coordinates needn't be ordered.
Matrices live next to explicit vectors. They are also functions; they take a row index
and a column index and they output a value at the given spot.
Neither the row indices nor the column vertices are required to form an ordered set.
That's why multiplication between matrices and vectors is defined only in structures
where addition forms a commutative semigroup. Consider the following example:
$$
\begin{pmatrix}
	1 & 2 & 3 \\
	4 & 5 & 6 \\
\end{pmatrix}
\*
\begin{pmatrix}
	7 \\ 8 \\ 9
\end{pmatrix}
=
\begin{pmatrix}
	? \\ \_
\end{pmatrix}
$$
We don't know whether the value at the question mark is equal to
$ (1 \* 7 + 2 \* 8) + 3 \* 9 $ or to
$ (2 \* 8 + 1 \* 7) + 3 \* 9 $ or to
any other ordering of summands.
This is why commutativity of addition is necessary for the definition to be valid.
On the other hand, we don't assume any property of multiplication in the
definition of multiplication between matrices and vectors; they don't even
have to be of the same type; we only require the elements of the vector
to have an action on the elements of the matrix (this is not a typo -- normally,
we would want matrices to have an action on vectors -- not in our work).

TODO formal definitions.

We start from bottom up!

\noindent
Theorem equalityFredholm is stated as follows:
\begin{lstlisting}
theorem equalityFredholm (A : Matrix I J F) (b : I → F) :
    (∃ x : J → F, A *ᵥ x = b) ≠ (∃ y : I → F, Aᵀ *ᵥ y = 0 ∧ b ⬝ᵥ y ≠ 0)
\end{lstlisting}
Theorem equalityFredholm\_lt is stated as follows:
\begin{lstlisting}
theorem equalityFredholm_lt (A : Matrix I J F) (b : I → F) :
    (∃ x : J → F, A *ᵥ x = b) ≠ (∃ y : I → F, Aᵀ *ᵥ y = 0 ∧ b ⬝ᵥ y < 0)
\end{lstlisting}
Theorem equalityFarkas is stated as follows:
\begin{lstlisting}
theorem equalityFarkas (A : Matrix I J F) (b : I → F) :
    (∃ x : J → F, 0 ≤ x ∧ A *ᵥ x = b) ≠ (∃ y : I → F, 0 ≤ Aᵀ *ᵥ y ∧ b ⬝ᵥ y < 0)
\end{lstlisting}
Theorem inequalityFarkas is stated as follows:
\begin{lstlisting}
theorem inequalityFarkas [DecidableEq I] (A : Matrix I J F) (b : I → F) :
    (∃ x : J → F, 0 ≤ x ∧ A *ᵥ x ≤ b) ≠ (∃ y : I → F, 0 ≤ y ∧ 0 ≤ Aᵀ *ᵥ y ∧ b ⬝ᵥ y < 0)
\end{lstlisting}
Theorem inequalityFarkas\_neg is stated as follows:
\begin{lstlisting}
theorem inequalityFarkas_neg [DecidableEq I] (A : Matrix I J F) (b : I → F) :
    (∃ x : J → F, 0 ≤ x ∧ A *ᵥ x ≤ b) ≠ (∃ y : I → F, 0 ≤ y ∧ -Aᵀ *ᵥ y ≤ 0 ∧ b ⬝ᵥ y < 0)
\end{lstlisting}
Theorem coordinateFarkas is stated as follows:
\begin{lstlisting}
theorem coordinateFarkas {I J : Type*} [Fintype J] [LinearOrderedDivisionRing R]
    (A : (I → R) →ₗ[R] J → R) (b : (I → R) →ₗ[R] R) :
    (∃ x : J → R, 0 ≤ x ∧ ∀ w : I → R, ∑ j : J, A w j • x j = b w) ≠ (∃ y : I → R, 0 ≤ A y ∧ b y < 0)
\end{lstlisting}
Theorem scalarFarkas is stated as follows:
\begin{lstlisting}
theorem scalarFarkas {J : Type*} [Fintype J] [LinearOrderedDivisionRing R] [AddCommGroup W] [Module R W]
    (A : W →ₗ[R] J → R) (b : W →ₗ[R] R) :
    (∃ x : J → R, 0 ≤ x ∧ ∀ w : W, ∑ j : J, A w j • x j = b w) ≠ (∃ y : W, 0 ≤ A y ∧ b y < 0)
\end{lstlisting}
Theorem fintypeFarkasBartl is stated as follows:
\begin{lstlisting}
theorem fintypeFarkasBartl {J : Type*} [Fintype J] [LinearOrderedDivisionRing R]
    [LinearOrderedAddCommGroup V] [Module R V] [PosSMulMono R V] [AddCommGroup W] [Module R W]
    (A : W →ₗ[R] J → R) (b : W →ₗ[R] V) :
    (∃ x : J → V, 0 ≤ x ∧ ∀ w : W, ∑ j : J, A w j • x j = b w) ≠ (∃ y : W, 0 ≤ A y ∧ b y < 0)
\end{lstlisting}
TODO.

Also TODO somehow politely say that the Mathlib's API for block matrices
is a mess and needs an overhaul.


\section {Proving the Farkas-Bartl theorem}

We prove finFarkasBartl and, in the end, we obtain fintypeFarkasBartl as corollary.

\medskip \noindent
\textbf{Theorem (finFarkasBartl):}
Let $n$ be a natural number.
Let $R$ be a linearly ordered division ring.
Let $W$ be an $R$-module.
Let $V$ be a linearly ordered $R$-module.
Let $A$ be an $R$-linear map from $W$ to $(\fin{n} \r R)$.
Let $b$ be an $R$-linear map from $W$ to $V$.
Exactly one of the following exists:
\begin{itemize}
\item nonnegative vector family $x : \fin{n} \r V$ such that, for all $w : W$, we have
$ \sum_{j : \fin{n}}\; (A~w)_j \bullet x_j = b~w $
\item vector $y : W$ such that $A~y \ge 0$ and $b~y < 0$
\end{itemize}
The only difference is that finFarkasBartl uses $\fin{n} = \{ 0, \dots, n\!-\!1 \}$
instead of an arbitrary (unordered) finite type $J$.

\medskip \noindent
Proof idea:
We first prove that both cannot exist at the same time.
Assume we have $x$ and $y$ of said properties.
We plug $y$ for $w$ and obtain
$ \sum_{j : \fin{n}}\; (A~y)_j \bullet x_j = b~y $.
On the left-hand side, we have a sum of nonnegative vectors,
which contradicts $b~y < 0$.
\smallskip

We prove ``at least one exists'' by induction on $n$.
If $n=0$ then $A~y \ge 0$ is a tautology.
We consider $b$. Either $b$ maps everything to the
zero vector, which allows $x$ to be the empty vector family,
or some $w$ gets mapped to a nonzero vector, where
we choose $y$ to be either $w$ or $(-w)$.
Since $V$ is linearly ordered, one of them satisfies $b~y<0$.
Now we precisely state how the induction step will be.

\medskip \noindent
\textbf{Lemma (industepFarkasBartl):}
Let $m$ be a natural number.
Let $R$ be a linearly ordered division ring.
Let $W$ be an $R$-module.
Let $V$ be a linearly ordered $R$-module.
Assume (induction hypothesis) that
for all $R$-linear maps $A_0 : W \r (\fin{m} \r R)$
and $b_0 : W \r V$, the formula
``$\forall y_0 : W \st A_0~y_0 \ge 0 \implies b_0~y_0 \ge 0$''
implies existence of a nonnegative vector family $x_0 : \fin{m} \r V$ such that,
for all $w_0 : W$, $ \sum_{i : \fin{m}}\; (A_0~w_0)_i \bullet (x_0)_i = b_0~w_0 $.
Let $A$ be an $R$-linear map from $W$ to $(\fin{m\!+\!1} \r R)$.
Let $b$ be an $R$-linear map from $W$ to~$V$.
Assume that, for all $y : W$, $\;A~y \ge 0$ implies $b~y \ge 0$.
We claim there exists a nonnegative vector family $x : \fin{m\!+\!1} \r V$
such that, for all $w : W$, we have
$ \sum_{i : \fin{m+1}}\; (A~w)_i \bullet x_i = b~w $.

\medskip \noindent
Proof idea:
Let $A_{<m}$ roughly mean $A \restriction \fin{m}$.
To be more precise, $A_{<m}$ is a function that maps $(w : W)$ to
$(A~w) \restriction \fin{m}$, so $A_{<m}$ is an $R$-linear map
from $W$ to $(\fin{m} \r R)$ that behaves exactly like $A$ where
it is defined.
We distinguish two cases. If, for all $y : W$, the inequality
$A_{<m}~y \ge 0$ implies $b~y \ge 0$, then plug $A_{<m}$
for $A_0$, obtain $x_0$, and construct a vector family $x$ such that
$x_m = 0$ and otherwise $x$ copies $x_0$. We easily check that
$x$ is nonnegative and that
$ \sum_{i : \fin{m+1}}\; (A~w)_i \bullet x_i = b~w $ holds.

In the second case, we have $y'$ such that $A_{<m}~y' \ge 0$
holds but $b~y' < 0$ also holds. We realize that $(A~y')_m < 0$.
We now declare $y := (A~y')_m \bullet y'$ and observe
$(A~y)_m = 1$. We establish the following facts (proofs are omitted):
\begin{itemize}
\item for all $w : W$, we have $A~(w - ((A~w)_m \bullet y)) = 0$
\item for all $w : W$, the inequality $A_{<m}~(w - ((A~w)_m \bullet y)) \ge 0$
implies $b~(w - ((A~w)_m \bullet y)) \ge 0$
\item for all $w : W$, the inequality $A_{<m}~w - A_{<m}~((A~w)_m \bullet y) \ge 0$
implies $b~w - b~((A~w)_m \bullet y) \ge 0$
\item for all $w : W$, the inequality $(A_{<m} - (z \mapsto (A~z)_m \bullet (A_{<m}~y)))~w \ge 0$
implies $(b - (z \mapsto (A~z)_m \bullet (b~y)))~w \ge 0$
\end{itemize}
We observe that
$A_0 := A_{<m} - (z \mapsto (A~z)_m \bullet (A_{<m}~y))$
and
$b_0 := b - (z \mapsto (A~z)_m \bullet (b~y))$
are $R$-linear maps.
Thanks to the last fact, we can apply induction hypothesis to $A_0$ and $b_0$.
We obtain a nonnegative vector family $x'$ such that,
for all $w_0 : W$, $ \sum_{i : \fin{m}} (A_0\;w_0)_i \bullet x'_i = b_0\;w_0 $.
It remains to construct a nonnegative vector family $x : \fin{m\!+\!1} \r V$
such that, for all $w : W$, we have
$ \sum_{i : \fin{m+1}}\; (A~w)_i \bullet x_i = b~w $.
We choose $x_m = b~y - \sum_{i : \fin{m}} (A_{<m}~y)_i \bullet x'_i$
and otherwise $x$ copies $x'$. We check that our $x$ has the required
properties. Qed.

\medskip
We complete the proof of finFarkasBartl by applying industepFarkasBartl
to $A_{\le n}$ and $b$. Finally, we obtain fintypeFarkasBartl from
finFarkasBartl using some boring mechanisms regarding equivalence between
finite types.


\section {Extended Farkas theorem}

\textbf{Theorem (extendedFarkas):}
Let $I$ and $J$ be finite types.
Let $F$ be a linearly ordered field.
Let $A$ be a matrix of type $(I \times J) \r F_\infty$.
Let $b$ be a vector of type $I \r F_\infty$.
Assume that $A$ does not have $\bot$ and $\top$ in the same row.
Assume that $A$ does not have $\bot$ and $\top$ in the same column.
Assume that $A$ does not have $\top$ in any row where $b$ has $\top$.
Assume that $A$ does not have $\bot$ in any row where $b$ has~$\bot$.
Exactly one of the following exists:
\begin{itemize}
\item nonnegative vector $x : J \r F$ such that $A \* x \le b$
\item nonnegative vector $y : I \r F$ such that $(-A^T) \* y \le 0$ and $b \* y < 0$
\end{itemize}
(restated)

\subsection{Proof idea}
We need to do the following steps in given order:
\begin{enumerate}
\item Delete all rows of both $A$ and $b$ where $A$ has $\bot$ or $b$ has $\top$
(they are tautologies).
\item Delete all columns of $A$ that contain $\top$
(they force respective variables to be zero).
\item If $b$ contains $\bot$, then $A \* x \le b$ cannot be satisfied,
but $y = 0$ satisfies $(-A^T) \* y \le 0$ and $b \* y < 0$. Stop here.
\item Assume there is no $\bot$ in $b$. Use inequalityFarkas\_neg.
In either case, extend $x$ or $y$ with zeros on	all deleted positions.
\end{enumerate}

\subsection{Counterexamples}

If $A$ has $\bot$ and $\top$ in the same row, it may happen that both $x$ and $y$ exist:
$$
A =
\begin{pmatrix}
	\bot & \top \\
	0 & -1
\end{pmatrix}
\qquad \qquad
b = \begin{pmatrix}	0 \\ -1 \end{pmatrix}
\qquad \qquad
x = \begin{pmatrix} 1 \\ 1 \end{pmatrix}
\qquad \qquad
y = \begin{pmatrix} 0 \\ 1 \end{pmatrix}
$$
If $A$ has $\bot$ and $\top$ in the same column, it may happen that both $x$ and $y$ exist:
$$
A =
\begin{pmatrix}
	\bot \\
	\top
\end{pmatrix}
\qquad \qquad
b = \begin{pmatrix}	-1 \\ 0 \end{pmatrix}
\qquad \qquad
x = \begin{pmatrix} 0 \end{pmatrix}
\qquad \qquad
y = \begin{pmatrix} 1 \\ 1 \end{pmatrix}
$$
If $A$ has $\top$ in a row where $b$ has $\top$, it may happen that both $x$ and $y$ exist:
%$x$ such that $A \* x \le b$ and 
%$y$ such that $(-A^T) \* y \le 0$ and $b \* y < 0$ exist:
$$
A =
\begin{pmatrix}
	\top \\
	-1
\end{pmatrix}
\qquad \qquad
b = \begin{pmatrix}	\top \\	-1 \end{pmatrix}
\qquad \qquad
x = \begin{pmatrix}	1 \end{pmatrix}
\qquad \qquad
y = \begin{pmatrix} 0 \\ 1 \end{pmatrix}
$$
If $A$ has $\bot$ in a row where $b$ has $\bot$, it may happen that both $x$ and $y$ exist:
$$
A =
\begin{pmatrix}
	\bot
\end{pmatrix}
\qquad \qquad
b = \begin{pmatrix}	\bot \end{pmatrix}
\qquad \qquad
x = \begin{pmatrix}	1 \end{pmatrix}
\qquad \qquad
y = \begin{pmatrix}	0 \end{pmatrix}
$$

\section {Proving the (extended) strong LP duality}

We note that it is intentional that both PRIM and DUAL are defined as
minimization problems.

TODO.


\section {Related work}

TODO presents an overcomplicated proof in Isabelle by analyzing
the Simplex algorithm that already had been formally verified.
It took them 30 pages to get to the basic Farkas; no generalization
was provided.

In Lean, it would be possible to prove Farkas for reals using the
Hahn-Banach separation theorem. However, we do not yet know that
the set of feasible solutions is closed.

TODO.


\section {Conclusion}

We formally verified several Farkas-like theorems in Lean 4.
We extended the existing theory to a new setting where some
coefficient can carry infinite values. We realized that the
abstract work with modules over linearly ordered division rings
and linear maps between them was fairly easy to carry on in
Lean 4 thanks to the library Mathlib that is perfectly suited
for such tasks. In contrast, manipulation with matrices got
tiresome whenever we needed a not-fully-standard operation.
It turns out Lean 4 cannot automate case analyses unless they
take place in the ``outer layers'' of formulas. Summation
over subtypes and summation of conditional expression made
us developed a lot of ad-hoc machinery which we would have
preferred to be handled by existing tactics. Another area
where Lean 4 is not yet helpful is the search for counterexamples.
Despite these difficulties, we find Lean 4 to be an extremely
valuable tool for elegant expressions of mathematical formulas
and for proving them formally.


\end{document}
